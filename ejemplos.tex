\chapter{Ejemplos de cálculos y visualizaciones}

\section{Integración de ecuaciones diferenciales ordinarias}

Veamos unas rutinas y herramientas para integrar ecuaciones diferenciales. Ocuparemos las técnicas básicas  que se ven en el curso de Física Computacional --los métodos de Euler y Runge--Kutta.

La ecuación que resolveremos es
\begin{equation}
\dot{\x} = \f(\x).
\end{equation}
Recordemos que eso es una notación para decir que
\begin{equation}
\dot{\x}(t) = \f(\x(t)).
\end{equation}
Es decir, si estamos en el punto $\x(t)$ al tiempo $t$, entonces la derivada instantánea en este momento y este punto del espacio fase está dada por $\f(\x(t))$.

Para resolver esta ecuación en la computadora, es necesario discretizarla de una manera u otra.
La manera más sencilla para hacerlo es discretizando el tiempo en pasos iguales de tamaño $\delta t$ y aproximando la derivada por una diferencia finita, o sea expandiendo en una serie de Taylor, lo cual da
\begin{equation}
 \x(t+\delta t) \simeq \x(t) + \delta t \dot{\x}(t) + O( (\delta t) ^2) = \x(t) + h \f(\x(t)) + O( (\delta t) ^2),
\end{equation}
donde $h = \delta t$ es el paso de tiempo.
Eso nos da el \defn{método de Euler}. Lo podemos implementar como una función que regresa el valor nuevo de la variable:
\begin{python}
def euler(t, x, h, f):
  return x + h*f(x)
\end{python}
Nótese que la función \inl{euler} toma \emph{el nombre de la función \inl{f} que integrar} como argumento.

Ahora para integrar la ecuación entre un tiempo initial $t_0$ y un tiempo final $t_f$, repetimos este paso muchas veces:
\begin{python}
def integrar(t0, tf, h, x0, f):
  lista_t = []
  lista_x = []
  
  x = x0
  for t in arange(t0, tf+h/2., h):
    x = euler(t, x, h, f)
    
    lista_t.append(t)
    lista_x.append(x)
    
  return lista_t, lista_x
\end{python}




