\chapter{Estructuras de control}

Por estructuras de control, se entiende los comandos tales como condicionales, bucles, y funciones, que cambian el orden de ejecución de un programa.  

Las estructuras de control tienen un formato en común. A diferencia de C++ y Fortran, no existen instrucciones que indiquen donde empieza y termina un bloque del programa: en Python, un bloque empieza por \inl{:}, y su extensión se indica por el uso de una cantidad definida de \defn{espacio en blanco} al principio de cada línea. Un buen editor de texto te permite hacer eso de manera automática.

\section{Condicionales}
Para checar una condición y ejecutar código dependiente del resultado, se ocupa \inl{if}:
\begin{python}
a = raw_input('Dame el valor de a')
if a > 0:
  print "a es positiva"
  a = a + 1
elif a == 0:		# NB: dos veces =
  print "a es 0"
else:
  print "a es negativa"
\end{python}
En el primer caso, hay dos comandos en el bloque que se ejecutará en el caso de que $a > 0$.
Nótese que \inl{elif} es una abreviación de \inl{else: if}

Las condiciones que se pueden ocupar incluyen: \inl{<}, \inl{<=} y \inl {!=} (no es igual).
Para combinar condiciones se ocupan las palabras \inl{and} y \inl{or}:
\begin{python}
a = 3; b=7
if a>1 and b>2:
  print "Grandes"
if a>0 or b>0:
  print "Al menos uno positivo"
\end{python}


\section{Bucles}
La parte central de muchos cálculos científicos consiste en llevar a cabo bucles, aunque en Python, como veremos, eso se desenfatiza.

\subsection{\inl{while}}
El tipo de bucle más sencillo es un \inl{while}, que ejecuta un bloque de código \defn{mientras} una cierta condición se satisfaga, y suele ocuparse para llevar a cabo una iteración en la cual no se conoce de antemano el número de iteraciones necesario. 

El ejemplo más sencillo de un \inl{while}, donde sí se conoce la condición terminal, para contar hasta 9, se puede hacer como sigue:
\begin{python}
i = 0
while i < 10:
  i += 1	# equivalente a i = i + 1
  print i
\end{python}
Nótese que si la condición deseada es del tipo ``hasta\ldots'', entonces se tiene que utilizar un \inl{while} con la condición opuesta. 

\ej Encuentra los números de Fibonacci menores que 1000.

\ej Implementa el llamado método babilónico para calcular la raiz cuadrada de un número dado $y$. Este método consiste en la iteración 
$x_{n+1} = \frac{1}{2} \left[x_n + (y / x_n) \right]$. >Qué tan rápido converge?



\subsection{\inl{for}}
Un bucle de tipo \inl{for} se suele ocupar para llevar a cabo un número de iteraciones que se conoce de antemano.
En el bucle de tipo \inl{for}, encontramos una diferencia importante con respecto a lenguajes más tradicionales: podemos iterar sobre \emph{cualquier} lista y ejecutar un bloque de código \defn{para} cada elemento de la lista:
\begin{python}
l = [1, 2.5, -3.71, "hola", [2, 3]]
for i in l:
  print 2*l
\end{python}

Si queremos iterar sobre muchos elementos, es más útil construir la lista. Por ejemplo, para hacer una iteración para todos los números hasta 100, podemos utilizar
\begin{python}
for i in range(100):
  print 2*i
\end{python}
>Qué es lo que hace la función \inl{range}? Para averiguarlo, lo investigamos de manera interactiva con \inl{ipython}:
\begin{python}
range(10)
range(3, 10, 2)
\end{python}
Nótese que \inl{range} no acepta argumentos de punto flotante.

\ej Toma una lista, y crea una nueva que contiene la duplicación de cada elemento de la lista anterior.

% \subsection{Cómo evitar bucles}
% Los bucles en Python suelen ser lentos, por ser un lenguaje interpretado. En ciertas circum

\section{Funciones}
Las funciones se pueden considerar como subprogramas que ejecutan una tarea dada. Corresponden a las subrutinas en Fortran. En Python, las funciones pueden o no aceptar argumentos, y pueden o no regresar resultados.

La sintaxis para declarar una función es como sigue:
\begin{python}
def f(x):
  print "Argumento x = ", x
  return x*x
\end{python}
y se llama así:
\begin{python}
f(3)
f(3.5)
\end{python}

Las funciones se pueden utilizar con argumentos de \emph{cualquier} tipo --el tipo de los argumentos nunca se especifica. Si las operaciones llevadas a cabo no se permiten para el tipo que se provee, entonces Python regresa un error:
\begin{python}
f("hola")
\end{python}

Las funciones pueden regresar varios resultados al juntarlos en un tuple:
\begin{python}
def f(x, y):
  return 2*x, 3*y
\end{python}

Se puede proporcionar una forma sencilla de documentación de una función al proporcionar un ``docstring":
\begin{python}
def cuad(x):
  """Funcion para llevar un numero al cuadrado.
  Funciona siempre y cuando el tipo de x permite multiplicacion.
  """
  return x*x
\end{python}
Ahora si interrogamos al objeto cuad con \inl{ipython}:
\begin{python}
cuad?
\end{python}
nos da esta información. Si la función está definida en un archivo, entonces
\inl{cuad??} muestra el código de la definición de la función.

\ej Pon el código para calcular la raiz cuadrada de un número adentro de una función. Calcula la raíz de los números de 0 hasta 10 en pasos de 0.2 e imprimir los resultados.

\section{Redirección de la salida}
En Linux, la manera más fácil (pero no muy flexible) de guardar datos en un archivo es utilizando la llamada ``redirección'' que provee el shell (Bash). Al correr un programa así desde Bash:
\begin{verbatim}
./programa > salida.dat
\end{verbatim}
la salida estándar (lo que aparece en la pantalla al correr el programa) se manda al archivo especificado. 
Así que
\begin{python}
python prog.py > salida.dat
\end{python}
mandará la salida del programa de python \inl{prog.py} al archivo \inl{salida.dat}.

\ej Guarda los resultados de las raices cuadradas, y grafícalos con \inl{gnuplot}:
\texttt{plot "salida.dat"}





\section{Bibliotecas matemáticas}
Para llevar a cabo cálculos con funciones más avanzadas, es necesario \emph{importar} la biblioteca estándar de matemáticas:
\begin{python}
from math import *
\end{python}
Eso no es necesario al correr \texttt{ipython -pylab}, que invoca un modo específico de \texttt{ipython}\footnote{Yo encuentro conveniente declarar un nuevo comando \texttt{pylab} agregando la línea \\ \texttt{alias pylab='ipython -pylab'} en el archivo \texttt{.bashrc} en mi directorio HOME.}.
Sin embargo, eso importa mucho más que la simple biblioteca de matemáticas.  Nótese que Python está escrito en C, así que la biblioteca \inl{math} provee acceso a las funciones matemáticas que están en la biblioteca estándar de C. Más adelante veremos como acceder a otras funciones, tales como funciones especiales.

\ej >Cuál es la respuesta de la entrada \inl{sqrt(-1)}?

\ej Intenta resolver la ecuación cuadrática $a x^2 + b x + c = 0$ para distintos valores de $a$, $b$ y $c$. 



Para permitir operaciones con números complejos como posibles respuestas, se ocupa la biblioteca (\defn{módulo}) \inl{cmath}.
Si hacemos
\begin{python}
from cmath import *
\end{python}
entonces se importan las funciones adecuadas:
\begin{python}
sqrt(-1)
\end{python}
Sin embargo, ya se ``perdieron'' las funciones anteriores, es decir, las nuevas funciones han reemplazado a las distintas funciones con los mismos nombres que había antes.

Una solución es importar la biblioteca de otra manera:
\begin{python}
import cmath
\end{python}
Ahora las funciones que se han importado de \inl{cmath} tienen nombres que empiezan por \inl{cmath.}:
\begin{python}
cmath.sqrt(-1)
\end{python}
\inl{ipython} ahora nos proporciona la lista de funciones adentro del módulo si hacemos \inl{cmath.<TAB>}.






