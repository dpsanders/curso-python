\chapter{Animaciones sencillas con Visual Python}

En este capítulo, veremos un paquete, Visual Python, que permite hacer animaciones en 3 dimensiones, en tiempo real, de una manera sencillísima. <Casi vale la pena aprender Python solamente para eso! Y evidentemente es excelente para enseñar conceptos básicos de la física. De hecho, fue diseñado principalmente para este fin.

\section{El paquete Visual Python}
La biblioteca de Visual Python (de aquí en adelante, ``Visual'') se carga con
\begin{python}
from visual import *
\end{python}
La manera más fácil de entender cómo utilizarlo es por ejemplo.

Empecemos creando una esfera:
\begin{python}
s = sphere()
\end{python}
Podemos ver \inl{sphere()} como una función que llamamos para crear un objeto tipo esfera.
Para poder manipular este objeto después, se lo asignamos el nombre \inl{s}.

Al crear objetos en Visual Python, \defn{se despliegan por automático}, y <en 3 dimensiones!
Como se puede imaginar, hay una cantidad feroz de trabajo abajo que permite que funcione eso, pero afortunadamente no tenemos que preocuparnos por eso, y simplemente podemos aprovechar su existencia.

>Qué podemos hacer con la esfera \inl{s}? Como siempre, \inl{ipython} nos permite averiguarlo al poner \inl{s.<TAB>}.
Básicamente, podemos cambiar sus \defn{propiedades internas}, tales como su color, su radio y su posición:
\begin{python}
s.color = color.red	# o  s.color = 1, 0, 0
s.radius = 0.5
s.pos = 1, 0, 0
\end{python}
Aquí, \inl{s.pos} es un vector, también definido por Visual, como podemos ver al teclear \inl{type(s.pos)}.
Los vectores en Visual son diferentes de los que provee \inl{numpy}. Los de Visual siempre tienen 3 componentes.

Ahora podemos construir otros objetos, incluyendo a \inl{box}, \inl{cylinder}, etc.
Nótese que la gráfica se puede rotar en 3 dimensiones con el ratón, al arrastar con el botón de derecho puesto, y se puede hacer un acercamiento con el botón central.

\section{Animaciones}
Ahora llega lo bueno. >Cómo podemos hacer una animación? Una animación no es más que una secuencia de imágenes, desplegadas rápidamente una tras otra. Así que eso es lo que tenemos que hacer:
\begin{python}
s = sphere()
b = box()
for i in range(10000):
  rate(100)
  s.pos = i/1000., 0, 0
\end{python}

\section{Agregando propiedades a objetos}
Supongamos que queremos pensar en nuestra esfera como una pelota. Entonces la pelota tendrá no solamente una posición, sino también una velocidad.  Se lo podemos crear así:
\begin{python}
pelota = sphere()
pelota.vel = vector(1,0,0)
\end{python}
Nótese que se tiene que poner explícitamente \inl{vector}, ya que sino sería una $n$-ada (tupla).
Ahora quedó definida la velocidad de la pelota como otra propiedad interna.


Eso es básicamente todo lo que hay que saber de Visual Python. También es posible interactuar con el teclado, extraer las coordenadas del ratón, etc.






