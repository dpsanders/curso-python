\chapter{Comunicándose con otros programas}

En este capítulo, veremos cómo un programa de Python puede interactuar con otros programas.

\section{Redirección}

En esta sección veremos algunas técnicas útiles del shell \inl{bash} que no están restringidos a \inl{python}.

Ya hemos visto que la manera más fácil de guardar información en un archivo es con la redirección de la salida estándar del programa: corriendo desde la terminal, ponemos
\begin{python}
python prog.py > salida.dat
\end{python}
y así creamos un archivo \inl{salida.dat}, donde se capta la información que anteriormente se mandaba a la terminal.

Lo mismo se puede hacer con la entrada estándar: si tenemos un programa que lee información que teclea el usuario:
\begin{python}
a = raw_input("Dame a: ")
b = raw_input("Dame b: ")
\end{python}
Entonces podemos también mandarle valores de un archivo, al redirigir la entrada estándar:
\begin{python}
python prog.py < datos.dat
\end{python}
donde \inl{datos.dat} contiene la información deseada.  A su vez, la salida de este programa se puede mandar a otro archivo con \inl{>}.

Finalmente, si queremos conectar la salida de un programa con la entrada de otro, utilizamos un ``tubo'' (``pipe''):
\begin{python}
python prog1.py | python prog2.py
\end{python}
Eso es equivalente a hacer
\begin{python}
python prog1.py > temp.dat
python prog2.py < temp.dat
\end{python}
pero sin la necesidad de crear el archivo temporal. 

Los pipes son muy útiles en unix, y se pueden encadenar.




\section{Argumentos de la línea de comandos}

Una manera más flexible y más útil para mandar información a un programa es a través de argumentos que se ponen en la mera línea de comandos. A menudo queremos
mandar a un programa los valores de un parámetro, para después poder hacer barridos del mismo.

Para hacerlo, Python provee una variable llamada \inl{argv}, que viene definida en el módulo \inl{sys}. Podemos ver qué es lo que contiene esta variable al correr un programa sencillo:
\begin{python}
from sys import argv

print "Argumentos: argv"
\end{python}
Si ahora llamamos a nuestro programa con
\begin{python}
python prog.py 10 0.5 ising
\end{python}
entonces vemos que \inl{argv} es una lista de cadenas, cada una representando uno de los argumentos, empezando por el nombre del script.
Por lo tanto, podemos extraer la información deseada con las herramientas normales de Python, por ejemplo
\begin{python}
from sys import argv
T, h = map(float, argv[1:3])
modelo = argv[3]
\end{python}

Sin embargo, si no damos suficientes argumentos, entonces el programa fracasará. Podemos atrapar una \defn{excepción} de este tipo con un bloque \inl{try}\ldots\inl{except}:
\begin{python}
from sys import argv, exit
try:
  T, h = map(float, argv[1:3])
  modelo = argv[3]
except:
  print "Sintaxis: python prog.py T h modelo"
  exit(1)
\end{python}
Es útil proveerle al usuario información acerca de la razón por la cual fracasó el programa. Aquí hemos utilizado la función \inl{exit} que viene también en el módulo \inl{sys}, para que salga del programa al encontrar un problema, pero eso no es obligatorio --podría poner valores por defecto de los parámetros en este caso, por ejemplo.


\section{Llamando a otros programas}

El módulo \inl{os} provee unas herramientas para mandar llamar a otros programas ``hijos'' desde un programa ``padre''\footnote{Hoy en día, se recomienda más bien el paquete \inl{subprocess} para esta funcionalidad, pero es más complicado.}.

La función más útil es \inl{system}, que permite correr un comando a través de un nuevo \inl{bash}; podemos enviar cualquier comando que funciona en \inl{bash}, en particular podemos correr otros programas. La función acepta una cadena:
\begin{python}
from os import system
system("ls")
\end{python}

Ahora podemos empezar a hacer cosas interesantes al construir los comandos con la sustitución de variables. Por ejemplo, si tenemos un programa \inl{prog.py} que acepta un argumento de la línea de comandos, podemos correrlo de manera consecutiva con distintos valores del parámetro con
\begin{python}
for T in range(10):
  comando = "python prog.py %g" % T
  system(comando)
\end{python}

Si queremos abrir un programa que se conectará al nuestro y recibirá comandos, utilizamos \inl{popen} para abrir un \emph{pipe}:
\begin{python}
from os import popen
gp = popen("gnuplot -persist", "w")
gp.write("plot sin(x)\n")
gp.close()
\end{python}


\section{Ejemplos}

\begin{itemize}
 \item 
Cartas de aceptación en \LaTeX

\item 
GIF animado con gnuplot

\item 
Figuras para distintos parámetros
\end{itemize}
