\chapter{Comunicándose con otros programas}

En este capítulo, veremos cómo un programa de Python puede interactuar con otros programas.

\section{Redirección}

En esta sección veremos algunas técnicas útiles del shell \inl{bash} que no están restringidos a \inl{python}.

Ya hemos visto que la manera más fácil de guardar información en un archivo es con la redirección de la salida estándar del programa: corriendo desde la terminal, ponemos
\begin{python}
python prog.py > salida.dat
\end{python}
y así creamos un archivo \inl{salida.dat}, donde se capta la información que anteriormente se mandaba a la terminal.

Lo mismo se puede hacer con la entrada estándar: si tenemos un programa que lee información que teclea el usuario:
\begin{python}
a = raw_input("Dame a: ")
b = raw_input("Dame b: ")
\end{python}
Entonces podemos también mandarle valores de un archivo, al redirigir la entrada estándar:
\begin{python}
python prog.py < datos.dat
\end{python}
donde \inl{datos.dat} contiene la información deseada.  A su vez, la salida de este programa se puede mandar a otro archivo con \inl{>}.

Finalmente, si queremos conectar la salida de un programa con la entrada de otro, utilizamos un ``tubo'' (``pipe''):
\begin{python}
python prog1.py | python prog2.py
\end{python}
Eso es equivalente a hacer
\begin{python}
python prog1.py > temp.dat
python prog2.py < temp.dat
\end{python}
pero sin la necesidad de crear el archivo temporal. 

Los pipes son muy útiles en unix, y se pueden encadenar.




\section{Argumentos de la línea de comandos}

Una manera más flexible y más útil para mandar información a un programa es a través de argumentos que se ponen en la mera línea de comandos. A menudo queremos
mandar a un programa los valores de un parámetro, para después poder hacer barridos del mismo.

Para hacerlo, Python provee una variable llamada \inl{argv}, que viene definida en el módulo \inl{sys}. Podemos ver qué es lo que contiene esta variable al correr un programa sencillo:
\begin{python}
from sys import argv

print "Argumentos: argv"
\end{python}
Si ahora llamamos a nuestro programa con
\begin{python}
python prog.py 10 0.5 ising
\end{python}
entonces vemos que \inl{argv} es una lista de cadenas, cada una representando uno de los argumentos, empezando por el nombre del script.
Por lo tanto, podemos extraer la información deseada con las herramientas normales de Python, por ejemplo
\begin{python}
from sys import argv
T, h = map(float, argv[1:3])
modelo = argv[3]
\end{python}

Sin embargo, si no damos suficientes argumentos, entonces el programa fracasará. Podemos atrapar una \defn{excepción} de este tipo con un bloque \inl{try}\ldots\inl{except}:
\begin{python}
from sys import argv, exit
try:
  T, h = map(float, argv[1:3])
  modelo = argv[3]
except:
  print "Sintaxis: python prog.py T h modelo"
  exit(1)
\end{python}
Es útil proveerle al usuario información acerca de la razón por la cual fracasó el programa. Aquí hemos utilizado la función \inl{exit} que viene también en el módulo \inl{sys}, para que salga del programa al encontrar un problema, pero eso no es obligatorio --podría poner valores por defecto de los parámetros en este caso, por ejemplo.


\section{Llamando a otros programas}

El módulo \inl{os} provee unas herramientas para mandar llamar a otros programas ``hijos'' desde un programa ``padre''\footnote{Hoy en día, se recomienda más bien el paquete \inl{subprocess} para esta funcionalidad, pero es más complicado.}.

La función más útil es \inl{system}, que permite correr un comando a través de un nuevo \inl{bash}; podemos enviar cualquier comando que funciona en \inl{bash}, en particular podemos correr otros programas. La función acepta una cadena:
\begin{python}
from os import system
system("ls")
\end{python}

Ahora podemos empezar a hacer cosas interesantes al construir los comandos con la sustitución de variables. Por ejemplo, si tenemos un programa \inl{prog.py} que acepta un argumento de la línea de comandos, podemos correrlo de manera consecutiva con distintos valores del parámetro con
\begin{python}
for T in range(10):
  comando = "python prog.py %g" % T
  system(comando)
\end{python}

Si queremos abrir un programa que se conectará al nuestro y recibirá comandos, utilizamos \inl{popen} para abrir un \emph{pipe}:
\begin{python}
from os import popen
gp = popen("gnuplot -persist", "w")
gp.write("plot sin(x)\n")
gp.close()
\end{python}

Una manera fácil de hacer una animación es utilizar la terminal de GIF animado en \inl{gnuplot}. Para utilizar esta terminal, es necesario contar con una versión de \inl{gnuplot} compilada de manera adecuada, con la librería \inl{gd}.

En la terminal de GIF animado, cada vez que uno ejecuta un \inl{plot}, otro cuadro se agrega a la animación. Por ejemplo:
\begin{python}
from os import popen
from numpy import arange
gp = popen("gnuplot", "w")
gp.write("set term gif anim\n")
gp.write("set out 'sin.gif'\n")	# archivo de salida de gnuplot

for a in arange(0., 10., 0.1):
  gp.write("plot sin(x + %g)\n title 'sin(x+%g)'\n" % (a, a) )
gp.write("set out\n")	# cerrar el archivo de salida de gnuplot
gp.close()
\end{python}


\section{Ejemplos}

Veamos algunos ejemplos de lo que podemos hacer utilizando Python para controlar otros programas. 
La sustitución de variables juega un papel crucial.

\subsection{Generando archivos sencillos con \LaTeX}
Empecemos con una aplicación sencilla, en la cual utilizamos Python para crear un archivo de \LaTeX\ para una carta de aceptación de un congreso, para la cual necesitamos el nombre del autor y el título del cartel.

Primero, para generar un archivo de \LaTeX, necesitamos poder manejar ciertos caracteres especiales en las cadenas. Para hacerlo, podemos poner \inl{r} al principio de la cadena para formar una cadena ``raw'' (``crudo''); Python automáticamente la convertirá en una cadena normal:
\begin{python}
s = r"\begin{document}"       
s
print s
\end{python}
Además, Python nos permite declarar cadenas con varias líneas, al utilizar tres comillas. Así que un archivo de \LaTeX\ sencillo se puede crear con
\begin{python}
codigo_latex = r"""
\documentclass{article}
\usepackage{mathptmx}	% fuente Times
\begin{document}
<Hola, David!
\end{document}
"""
\end{python}
Ahora tenemos que guardar eso en un archivo, y compilarlo
\begin{python}
from os import system

archivo = open("temp.tex", "w")
archivo.write(codigo_latex)	# escribir contenido de variable en archivo
archivo.close()

comando = "pdflatex temp.tex"	# comando para compilar el archivo .tex
system(comando)	# ejecutar el comando

comando = "acroread temp.pdf"
system(comando)	# ver el archivo que creamos
\end{python}

Ahora está fácil reemplazar el nombre con el contenido de una variable, por ejemplo una que leemos de la línea de comandos, con el siguiente código que guardamos como \inl{saludo.py}:
\begin{python}
from sys import argv, exit

try:
  nombre = argv[1]
except:
  print "Sintaxis: python saludo.py nombre"
  exit(1)

codigo_latex = r"""
\documentclass{article}
\usepackage{mathptmx}	% fuente Times
\begin{document}
<Hola, %s!
\end{document}
""" % nombre
\end{python}
Lo guardamos y compilamos como antes. Nótese que la sustitución de variables también se puede utilizar con las cadenas crudas y las con múltiples líneas.

Incluso podríamos leer el código \LaTeX\ desde un archivo. En el archivo ponemos \inl{%s} etc.\ donde queremos sustituir variables. Leemos el contenido del archivo en una cadena, que entonces contendrá explícitamente los \inl{%s}, y \emph{luego} hacemos la sustitución. Así, si temp.tex contiene 
\begin{python}
\documentclass{article}
\usepackage{mathptmx}	% fuente Times
\begin{document}
<Hola, %s!
\end{document}
\end{python}
podemos hacer
\begin{python}
entrada = open("temp.tex", "r")
codigo_latex = entrada.read()	# leer todo el contenido en una cadena
nombre = "David"
codigo_latex = codigo_latex % nombre
\end{python}
En la última línea, reemplazamos el contenido de \inl{codigo_latex} con el resultado de hacer la sustitución de variables.

\subsection{Gráficas del comportamiento de un sistema para distintos valores de un parámetro}
Ahora veamos un ejemplo más complicado, también ocupando \LaTeX, donde crearemos un archivo con varias figuras que retratan el comportamiento de un sistema con distintos valores de un parámetro. Esta idea se tomó del excelente libro de Langtangen. Ahí se dan varias variaciones sobre el tema, incluyendo cómo generar un reporte en HTML.

Lo primero que hay que hacer es generar las figuras. Esto se puede hacer por ejemplo con \inl{gnuplot}, pero aquí emplearemos \inl{pylab}. Para enfatizar la idea, nuestro ``sistema'' será una gráfica de $\sin(kx)$ para distintos $k$, pero basta con reemplazar la gráfica con datos reales.
Primero generemos una gráfica:
\begin{python}
from pylab import *

t = arange(-2*pi, 2*pi, 0.1)
plot(t, sin(t), '-o')
savefig("sin.pdf")
\end{python}

Ahora repitamos eso para generar varias figuras:
\begin{python}
from pylab import *

t = arange(-2*pi, 2*pi, 0.1)

for k in range(1, 7):
  plot(t, sin(k*t), '-o')
  xlabel("$x$")
  ylabel("$sin(%dx)$" % k)
  savefig("sin%d.pdf" % k)
\end{python}
Nótese el uso directo de la sustitución de variables para generar la cadena correspondiente al nombre del archivo \emph{al momento de utilizarla}. Acordémonos que \inl{pylab} puede ocupar directamente etiquetas hechas con \LaTeX.

Ahora agarremos estas figuras y pongámoslas en un solo PDF, a través de \LaTeX.
Para hacerlo, primero generemos el principio del archivo:
\begin{python}
latex = r"""
\documentclass{article}
\usepackage{graphicx}	% para incluir figuras
\begin{document}
\begin{figure}
"""
\end{python}

Ahora para cada figura, tenemos que agregar el comando de \LaTeX\ para importarla:
\begin{python}
for k in range(1, 7):
  latex += r"\includegraphics[scale=0.4]{sin%d}" % k + "\n"
\end{python}
Nótese que es necesario separar el \inl{\n} para que no aparezca adentro de la cadena cruda.
Finalmente hay que terminar la figura:
\begin{python}
latex += "\end{figure}"
\end{python}
Podemos checar que todo pasó bien con
\begin{python}
print latex
\end{python}


Una mejor manera de incluir las figuras es utilizando el paquete \inl{subfigure} de \LaTeX, que provee una manera bonita de incluir subfiguras. Después de incluirlo con \verb+\usepackage{subfigure}+, podemos utilizarlo adentro de entorno \inl{figure} como sigue: \verb+\subfigure[subpie]{\includegraphics{subfigura}}+, donde \inl{subpie} es el pie de figura para la subfigura, y \inl{subfigura} el archivo. Así que podemos poner:
\begin{python}
subfigura = r"""
\subfigure[$k=%d$] {
\includegraphics[scale=0.4]{sin%d}
}
"""	# cada figura se ve así

# Ahora incluir cada figura:
for k in range(1, 7):
  latex += subfigura % (k, k)
\end{python}




