\chapter{>Qué es Python?}

Python es un lenguaje de programación que surgió en 1990. Desde entonces se ha desarrollado enormemente, para volverse un lenguaje de programación 
moderno, y uno de los más utilizados en el mundo.

Python es un lenguaje \emph{interpretado} --no es necesario compilar constantemente cada programa, si no se puede utilizar de manera interactiva.
Por lo tanto, es una herramienta idónea para llevar a cabo tareas de cómputo científico, tales como análisis de datos, graficación de resultados, y exploración de problemas. En este sentido, se puede comparar con programas comerciales tales como Matlab. También hay paquetes que permiten llevar a cabo cálculos simbólicos a través de Python, de los cuales destaca \texttt{sage}  (\url{www.sagemath.org}), por lo cual también se puede considerar como competidor de Mathematica y Maple.

Python viene con una filosofía de ``baterías incluidas''. Esto quiere decir que hay muchas bibliotecas disponibles que están diseñadas para facilitarnos la vida al hacer diferentes tareas, desde el cómputo científico hasta la manipulación de páginas web. Por lo tanto, Python es un lenguaje sumamente versátil.

Cabe enfatizar que se suele encontrar Python fácil de aprenderse y de utilizarse. Por lo tanto, considero que Python es también el lenguaje idóneo para la enseñanza del cómputo científico en la Facultad de Ciencias, un campo que se ha vuelto de suma importancia.

\section{Meta del curso}
La meta principal de este curso es la de enseñar las técnicas básicas de Python en el contexto del cómputo científico, y en particular de la física computacional,
con un fin de actualización docente en la Facultad de Ciencias.