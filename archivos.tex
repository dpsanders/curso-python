\chapter{Cómo leer y escribir archivos}
Uno de los usos principales de Python para el cómputo científico es el de procesar datos generados en otro lado.
Por lo tanto, es necesario saber cómo importar y exportar archivos.




\section{Redirección de la salida}
En Linux, la manera más fácil (pero no muy flexible) de guardar datos en un archivo es utilizando la llamada ``redirección'' que provee el shell (Bash). Al correr un programa así desde Bash:
\begin{verbatim}
./programa > salida.dat
\end{verbatim}
la salida estándar (lo que aparece en la pantalla al correr el programa) se manda al archivo especificado. 
Así que
\begin{python}
python prog.py > salida.dat
\end{python}
mandará la salida del programa de python \inl{prog.py} al archivo \inl{salida.dat}.

\ej Guarda los resultados de las raices cuadradas de diferentes números, y grafícalos con \inl{gnuplot}:
\texttt{plot "salida.dat"}

\ej Haz una gráfica de la velocidad de convergencia del método Babilónico para calcular la raíz cuadrada.

\section{Leer archivos}
Para leer un archivo, primero es necesario abrirlo para leerse. Supongamos que tenemos un archivo llamado \inl{datos.dat}, entonces lo podemos abrir para su lectura con
\begin{python}
entrada = open("datos.dat", "r")
\end{python}
El segundo argumento, \inl{"r"}, es para indicar que se va a leer  (``read'') el archivo.
El objeto \inl{entrada} ahora representa el archivo.

Para leer del archivo abierto, hay varias posibilidades. Podemos leer todo de un golpe con \inl{entrada.read()}, leer todo por líneas con \inl{entrada.readlines()}, o línea por línea con \inl{entrada.readline()}. [Nótese que lo que se ha leído ya no se puede leer de nuevo sin cerrar el archivo con \inl{entrada.close()} y volverlo a abrir.]

Por ejemplo, podemos utilizar
\begin{python}
for linea in entrada.readlines():
  print linea
\end{python}
Sin embargo, tal vez la opción más fácil e intuitiva es
\begin{python}
for linea in entrada:
  print linea
\end{python}
Es decir, el archivo <se comporta como una secuencia!

Ahora, la línea viene como una sola cadena, con espacios etc. Para extraer la información, primero necesitamos dividirlo en palabras:
\begin{python}
palabras = linea.split()
\end{python}
Si todos los datos en realidad son números, entonces tenemos que procesar cada palabra, convirtiéndola en un número:
\begin{python}
datos = []
for i in palabras:
  datos.append( float(i) )
\end{python}

Resulta que hay una manera más fácil, de más alto nivel, y (!) más rápida de hacer eso:
\begin{python}
map(float, palabras)
\end{python}
Eso literalmente mapea la función \inl{float} sobre la lista \inl{palabras}.

Combinando todo, podemos escribir
\begin{python}
entrada = open("datos.dat", "r")

for linea in entrada:
  datos = map( float, linea.split() )
\end{python}
Ahora adentro del bucle podemos manipular los datos como queramos.

\ej Para un archivo con dos columnas, leer los datos de cada columna en dos listas por separado.


\section{Escribir archivos}

El método de redirigir la salida que vimos arriba es rápido y fácil. Sin embargo, no es de ninguna manera flexible, por ejemplo no 
podemos especificar desde nuestro programa el nombre del archivo de salida, ni escribir en dos archivos diferentes.

Para hacer eso, necesitamos poder abrir un archivo para escribirse:
\begin{python}
salida = open("resultados.dat", "w")
\end{python}

La parte más complicada viene al momento de escribir en el archivo. Para hacerlo, ocupamos la función \inl{salida.write()}.
Sin embargo, esta función puede escribir solamente \emph{cadenas}. Por lo tanto, si queremos escribir números contenidos en variables, es necesario convertirlos primero a la forma de cadena.

Una manera de hacer eso es con la función \inl{str()}, y concatenar distintas cadenas con \inl{+}:
\begin{python}
a = 3
s = "El valor de a es " + str(a)
\end{python}
Sin embargo, para largas secuencias de datos eso se vuelve latoso.

\section{Sustitución de variables en cadenas}
La manera más elegante de hacerlo es con la sustitución de variables en cadenas\footnote{Eso es muy parecido a \inl{printf} en C.}. En una cadena ponemos una secuencia especial de caracteres, empezando por \texttt{\%}, que indica que se sustituirá el valor de una variable:
\begin{python}
a = 3
s = "El valor de a es %d" % a
b = 3.5; c = 10.1
s = "b = %g;  c = %g" % (b, c)
\end{python}
El caracter despues del \texttt{\%} indica el tipo de variable que incluir en la cadena: \inl{d} corresponde a un entero, \inl{g} o \inl{f} a un flotante, y \inl{s} a otra cadena. También se puede poner un número entero, que es el tamaño del campo, y un punto decimal seguido por un número, que da el número de decimales para el caso de \inl{f}.

Finalmente ahora podemos imprimir en el archivo, por ejemplo
\begin{python}
salida.write("%g\t%g" % (a,b) )
\end{python}
Aquí, \inl{\t} representa un tabulador, y \inl{\n} una nueva línea.

 











