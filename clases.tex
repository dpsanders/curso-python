\chapter{Programación orientada a objetos: clases}

En este capítulo, veremos uno de los temas más revolucionarios en materia de programación: la \defn{programación orientada a objetos}. Aunque hemos estado utilizando los objetos de manera implícita a través del curso, ahora veremos cómo nosotros podemos definir nuestros propios tipos de objetos nuevos, y para qué sirve.

\section{La programación orientada a objetos: las clases}

Consideremos una situación clásica en el cómputo científico: un sistema que consiste en cierto número de partículas en 2 dimensiones, que tienen \defn{propiedades internas}, como una posición, una velocidad y una masa, y que se pueden mover con una regla tipo Euler.

Para una partícula, podríamos definir sus variables simplemente como sigue:
\begin{python}
x = y = 0.0    # posicion
vx =  vy = 1.0  # velocidad
\end{python}
Un paso de Euler de tamaño \inl{dt} se puede implementar con
\begin{python}
dt = 0.1
x += dt * vx;
y += dt * vy;
\end{python}


Pero ahora, >qué hacemos si necesitamos otra partícula más? Podríamos poner
\begin{python}
x1 = y1 = 0.0
x2 = y2 = 0.0
vx1 = vy1 = 1.0
vx1 = vy1 = 1.0
\end{python}
Si las partículas también tienen masas y colores, entonces se vuelve muy tedioso, ya que tenemos que actualizar todo dos veces para cada propiedad nueva:
\begin{lstlisting}
m1 = m2 = 0.0;
c1 = c2 = 0.0;
\end{lstlisting}
Para muchas particulas podríamos emplear arreglos (listas, en Python), que reduciría el trabajo.

Pero ahora podríamos imaginarnos que por alguna razón queremos  agregar otra partícula, o incluso duplicar toda la simulación.
  Entonces tendríamos que duplicar a mano todas las variables, cambiando a la vez sus nombres, lo cual
seguramente conducirá a introducir  errores.

Sin embargo, \emph{conceptualmente} tenemos muchas variables que están relacionadas: todas \emph{pertenecen} a una partícula.
Hasta ahora, no hay manera de expresar esto en el programa.

\section{La solución: objetos, a través de clases}

Lo que queremos hacer, entonces, es reunir todo lo que corresponde a una partícula en un \emph{nuevo tipo de objeto}, llamado \inl{Particula}.  Luego podremos decir que  \inl{p1} y \inl{p2} son \inl{Particula}s, 
o incluso hacer un arreglo (lista) de \inl{Particula}s.

Toda la información que le corresponde a una partícula dada estará contenida adentro del objeto; esta información formará parte del objeto no sólo
conceptualmente, sino también en la representación en el programa. 

Podemos pensar en un objeto, entonces, como un tipo de ``caja negra'', que tiene propiedades internas, y que puede interactuar de alguna manera con el mundo externo. No es necesario saber o entender qué es lo que hay adentro del objeto para entender cómo funciona. Se puede pensar que corresponde a una caja con palancas, botones y luces: las palancas y los botones proveen una manera de darle información o instrucciones a la caja, y las luces dan información de regreso al mundo externo.

Para implementar eso en Python, se declara una \defn{clase} llamada \inl{Particula} y se crea una \defn{instancia} de esta clase, llamada \inl{p}.
\begin{python}
class Particula:
  x = 0.0
  
p = Particula()
p.x
\end{python}
Nótese que \inl{p} tiene \emph{adentro} una propiedad, que se llama \inl{x}. Podemos interpretar \inl{p.x} como una \inl{x} que le \emph{pertenece} a  \inl{p}.

Si ahora hacemos
\begin{lstlisting}
p2 = Particula()
p2.x
\end{lstlisting}
entonces tenemos una variable completamente \emph{distinta} que también se llama \inl{x}, pero que ahora le pertenece a  \inl{p2}, que es otro objeto de tipo  \inl{Particula}. De hecho, más bien podemos pensar que \inl{p2.x} es una manera de escribir \inl{p2_x}, que es como lo hubiéramos podido escribir antes, que tiene la bondad de que ahora tambíen tiene sentido pensar en el conjunto \inl{p} como un todo.

Remarquemos que ya estamos acostumbrados a pensar en cajas de este tipo en matemáticas, al tratar con vectores, matrices, funciones, etc.

\section{Métodos de clases}

Hasta ahora, una clase actúa simplemente como una caja que contiene datos.  
Pero objetos no sólo tienen información, sino también pueden \emph{hacer cosas}.
Para hacerlo, también se pueden definir \emph{funciones} --llamadas \defn{métodos}-- que le pertenecen al objeto.  
Simplemente se definen las funciones adentro de la declaración de la clase:
\begin{python}
class Particula:
  x = 0.0
  v = 1.0
  
  def mover(self, dt):
    self.x += self.v*dt

p = Particula()
print p.x
p.mover(0.1)
print p.x
\end{python}

Nótese que los métodos de las clases siempre llevan un argumento extra, llamado \inl{self}, que quiere decir ``sí mismo''.
Las variables que  pertenecen a la clase, y que se utilizan adentro de estas funciones, llevan \inl{self.}, para indicar que son variables que forman parte de la clase, y no variables globales. Podemos pensar en las variables internas a la clase como variables ``pseudo-globales'', ya que están accesibles desde cualquier lugar \emph{adentro} de la clase.


\section{Funciones inicializadoras}
Cuando creamos una instancia de un objeto, muchas veces queremos \defn{inicializar} el objeto al mismo tiempo, es decir pasarle información sobre su estado inicial. En Python, esto se hace a través de una función inicializadora, como sigue:
\begin{python}
class Particula:
  def __init__(self, xx=0.0, vv=1.0):
    self.x = xx
    self.v = vv
  
  def mover(self, dt):
    self.x += self.v*dt

p1 = Particula()
print p1.x
p2 = Particula(2.5, 3.7)
print p2.x
p1.mover(0.1)
p2.mover(0.1)
print p1.x, p2.x
\end{python}
Nótese que la función inicializadora debe llamarse \inl{__init__}, con dos guiones bajos de cada lado del nombre \inl{init}.
Puede tomar argumentos que se utilizan para inicializar las variables de la clase.

\section{Métodos internos de las clases}
Las clases pueden ocupar varios métodos para proveer un interfaz más limpio para el usuario. Por ejemplo, al poner 
\inl{print p1} en el último ejemplo, sale algo así como
\begin{python}
<__main__.Particula instance at 0x2849cb0>
\end{python}
Podemos hacer que salga algo más útil al proveer adentro de la clase una función \inl{__str__}:
\begin{python}
class Particula:
  def __init__(self, xx=0.0, vv=1.0):
    self.x = xx
    self.v = vv

  def __str__(self):
    return "Particula(%g, %g)" % (self.x, self.v)

p1 = Particula()
print p1
\end{python}

También podemos definir funciones que permiten que podamos llevar a cabo operaciones aritméticas etc.\ con objetos de este tipo, es decir, podemos \defn{sobrecargar} los operadores para que funcionen con nuestro nuevo tipo. Eso es lo que pasa con los \inl{array} de \inl{numpy}, por ejemplo.

\ej Haz un objeto para representar a una partícula en 2D. Haz una nube de tales partículas.

\section{Herencia}
Las clases también pueden formar jerarquías al utilizar la \defn{herencia}, que quiere decir que una clase es un ``tipo de'', o ``subtipo de'' otro.
