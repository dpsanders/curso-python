\chapter{Paquetes para el cómputo científico en Python}

Este capítulo propone dar un breve resumen de algunos de los muchos paquetes disponibles para llevar a cabo distintas tareas de cómputo científico en Python.

\section{Cálculos numéricos con \texttt{scipy}}
El paquete \inl{scipy} provee una colección de herramientas para llevar a cabo tareas numéricas, como son los siguientes submódulos:
funciones especiales (\inl{special}); integración de funciones  y de ecuaciones diferenciales ordinarias (\inl{integrate}), optimización y raíces de funciones (\inl{optimize}), álgebra lineal (\inl{linalg}), incluyendo para matrices escasas (\inl{sparse}), y estadísticas (\inl{stats}).

Veamos algunos ejemplos. Para utilizar las funciones especiales, hacemos
\begin{python}
from scipy import special
for i in range(5):


special.gamma(3)
special.gamma(3.5)

x = arange(-2, 2, 0.05)

for i in range(5):
  plot(x, map(special.hermite(i), x))
\end{python}

Para integrar la ecuación de Airy (un ejemplo de la documentación de \inl{scipy})
\begin{equation}
\frac{d^2 w}{d z^2} - z w(z) = 0,
\end{equation}
podemos poner
\begin{python}
from scipy.integrate import odeint
from scipy.special import gamma, airy
y1_0 = 1.0/3**(2.0/3.0)/gamma(2.0/3.0)
y0_0 = -1.0/3**(1.0/3.0)/gamma(1.0/3.0)
y0 = [y0_0, y1_0]

def func(y, t):
  return [t*y[1],y[0]]

def gradiente(y,t):
  return [[0,t],[1,0]]

x = arange(0,4.0, 0.01)
t = x
ychk = airy(x)[0]
y = odeint(func, y0, t)
y2 = odeint(func, y0, t, Dfun=gradient)

print ychk[:36:6]
print y[:36:6,1]
print y2[:36:6,1]
\end{python}
En la segunda llamada a \inl{odeint}, mandamos explícitamente la función que calcula la derivada (gradiente) de \inl{f}; en la primera llamada, no fue necesario contar con una función que calculara eso.

Encontrar raíces de funciones es más o menos sencillo:
\begin{python}
def f(x):
  return x + 2*cos(x)

def g(x):
  out = [x[0]*cos(x[1]) - 4]	
  out.append(x[1]*x[0] - x[1] - 5)
  return out

from scipy.optimize import fsolve
x0 = fsolve(func, 0.3)
x02 = fsolve(func2, [1, 1])
\end{python}


\section{Cálculos simbólicos con \texttt{sympy}}
El módulo \inl{sympy} provee una colección de rutinas para hacer cálculos simbólicos.
Las variables se tienen que declarar como tal, y luego se pueden utilizar en expresiones simbólicas:
\begin{python}
from sympy import *
x, y, z = symbols('xyz')
k, m, n = symbols('kmn', integer=True)
f = Function("f")

x + x
(x + y) ** 2
z = _
z.expand()

z.subs(x, 1)

limit(sin(x) / x, x, 0)
limit(1 - 1/x, x, oo)

diff(sin(2*x), x)
diff(sin(2*x), x, 3)

cos(x).series(x, 0, 10)

integrate(log(x), x)
integrate(sin(x), (x, 0, pi/2))

f(x).diff(x, x) + f(x)
dsolve(f(x).diff(x, x) + f(x), f(x))

solve(x**4 - 1, x)

pi.evalf(50)
\end{python}


\section{\texttt{mayavi2}}

 El paquete \inl{mayavi2} provee una manera de visualizar conjuntos de datos complejos en 3D. Para utilizarlo de manera interactiva, comenzamos con
\begin{python}
ipython -wthread

import numpy as np

def V(x, y, z):
    """ A 3D sinusoidal lattice with a parabolic confinement. """
    return np.cos(10*x) + np.cos(10*y) + np.cos(10*z) + 2*(x**2 + y**2 + z**2)

X, Y, Z = np.mgrid[-2:2:100j, -2:2:100j, -2:2:100j]
V(X, Y, Z)

from numpy import *
from enthought.mayavi import mlab

x, y, z = random.rand(3, 10)
mlab.points3d(x, y, z)
c = random.rand(10)
mlab.points3d(x, y, z, color=c)

mlab.clf()

mlab.contour3d(X, Y, Z, V)


\end{python}


\section{Entorno completo: \texttt{sage}}
El paquete \inl{sage}, disponible libremente de la página \url{http://www.sagemath.org}, pretende proveer una manera de reemplazar a programas del estilo de Mathematica, al proveer un entorno completo para hacer cálculos matemáticos. Es un entorno que provee un interfaz tipo Python a muchos paquetes libres para hacer matemáticas.  

También vien con un interfaz disponible por el internet para interactuar con los ``notebooks''.
Al bajar e iniciar \inl{sage}, provee solamente el interfaz tipo línea de comandos. El comando \inl{notebook()} corre este interfaz gráfico, a lo cual conecta uno a través de su navegador (browser). Este interfaz igual se puede utilizar de manera remota para conectarse a un servidor de \inl{sage}, por ejemplo el que se puede utilizar libremente en la dirección  






