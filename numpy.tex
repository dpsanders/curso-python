\chapter{Operaciones matemáticas y la biblioteca de arreglos \texttt{numpy}}

\section{Conversión de números}
Acordémonos que hay distintos tipos de números en Python, principalmente enteros (\inl{int}) y flotantes (\inl{float}).
Para convertir entre diferentes tipos, incluyendo cadenas, podemos utilizar
\begin{python}
a = float(3)
b = float('3.5')
\end{python}


\section{Aritmética con precisión arbitraria}

A veces, es necesaria poder llevar a cabo operaciones aritméticas con números flotantes (reales) con precisión superior a los 16 dígitos que provee el \inl{float} (número de ``doble precisión'') de Python. Para hacerlo, existen varios proyectos que proveen bibliotecas con este fin.
La mayoría de estas bibliotecas son interfaces a otros proyectos escritos en C++.

Aquí veremos una opción, la biblioteca \inl{mpmath}, que está escrito completamente en Python. En principio eso lo hace más lento, pero más fácil de entender y modificar el código.

Para cargar la biblioteca, hacemos
\begin{python}
from mpmath import *
\end{python}
Para cambiar la precisión, hacemos
\begin{python}
mp.dps = 50
\end{python}
Ahora, para crear un número flotante de esta precisión, hacemos
\begin{python}
x = mpf('1.0')
\end{python}
donde el número se expresa como cadena.

Al hacer manipulaciones con \inl{x}, los cálculos se llevan a cabo en precisión múltiple.
Por ejemplo,
\begin{python}
print x/6., x*10
print mpf('2.0')**2**2**2**2
\end{python}
Con \inl{mpmath}, no hay límite del exponente que se puede manejar.
También están definidas muchas funciones, por ejemplo \inl{sin}, \inl{exp} y \inl{log}.

Para imprimir un número con una precisión dada, usamos
\begin{python}
nprint(x, 20)
\end{python}

\section{La biblioteca \texttt{numpy}}
Hasta ahora, hemos utilizado listas para guardar y manipular datos. Sin embargo, las listas no se comportan como vectores, y menos como matrices --al sumarlos no se comportan de la manera adecuada, etc. El propósito de la biblioteca \inl{numpy} es justamente el de proporcionar objetos que represantan a vectores y matrices matemáticos, con todas las bondades que traen consigo este tipo de objetos.

La biblioteca se carga con
\begin{python}
from numpy import *
\end{python}
Provee muchas funciones para trabajar con vectores y matrices.

\section{Creando vectores}
Los vectores se llaman (aunque es un poco confuso) \inl{array}s, y se pueden crear de distintas maneras. 
Son como listas, pero con propiedades y métodos adicionales para funcionar como objetos matemáticos.
La manera más general de crear un vector es justamente convirtiendo desde una lista de números:
\begin{python}
from numpy import *

a = array( [1, 2, -1, 100] )
\end{python}
Un vector de este tipo puede contener sólo un tipo de objetos, a diferencia de una lista normal de Python.
Si todos los números en la lista son enteros, entonces el tipo del arreglo también lo es, como se puede comprobar con \inl{a.dtype}.

Es común querer crear vectores de cierto tamaño con todos ceros:
\begin{python}
b = zeros( 10 )
print b
\end{python}
o todos unos:
\begin{python}
b = ones( 10 )
print b
\end{python}

También hay distintas maneras de crear vectores que consisten en rangos ordenados, por ejemplo \inl{arange}, que funciona como \inl{range}, con un punto inicial, un punto final, y un paso:
\begin{python}
a = arange(0., 10., 0.1)
\end{python}
y \inl{linspace}, donde se especifica puntos iniciales y finales y un número de entradas:
\begin{python}
l = linspace(0., 10., 11)
\end{python}

Una notación abreviada para construir vectores es \inl{r_}, que se puede pensar como una abreviación de ``vector \defn{r}englón'':
\begin{python}
a = r_[1,2,10,-1.]
\end{python}
Este método se extiende para dar una manera rápida de construir rangos:
\begin{python}
r_[3:7]
r_[3:7:0.5]
r_[3:7:10j]
\end{python}
Este último utiliza un ``número complejo'' simplemente como otra notación, y es el equivalente de \inl{linspace(3,7,10)}.




\section{Operando con vectores}

Los vectores creados de esta manera se pueden sumar, restar etc., como si fueran vectores matemáticos. Todas las operaciones se llevan a cabo entrada por entrada:
\begin{python}
a = array( [1., 4., 7. ])
b = array( [1., 2., -2. ])
print a+b, a-b, a*b, a/b, a**b
\end{python}

\ej >Cómo se puede crear un vector de 100 veces $-3$?

Las funciones más comunes entre vectores ya están definidas en \inl{numpy}, entre las cuales se encuentran \inl{dot(a, b)} para productos escalares de dos vectores de la mismo longitud, y \inl{cross(a, b)} para el producto cruz de dos vectores de longitud $3$.

Además, cualquier función matemática como \inl{sin} y \inl{exp} se puede aplicar directamente a un vector, y regresará un vector compuesto por esta función aplicada a cada entrada del vector, tipo \inl{map}.

Es más: al definir una función el usuario, esta función normalmente también se pueden aplicar directamente a un vector:
\begin{python}
def gauss(x):
  return 1./ (sqrt(2.)) * exp(-x*x / 2.)

gauss( r_[0:10] )
\end{python}

\section{Extrayendo partes de un vector}
Para extraer subpartes de un vector, la misma sintaxis funciona como para listas: se extraen componentes (entradas) individuales con
\begin{python}
a = array([0, 1, 2, 3])
print a[0], a[2]
\end{python}
y subvectores con
\begin{python}
b = a[1:3]
\end{python}
Nótese, sin embargo, que en este caso la variable \inl{b} \emph{no} es una \emph{copia} de esta parte de \inl{a}. Más bien, es una \defn{vista} de \inl{a}, así que ahora si hacemos
\begin{python}
b[1] = 10
\end{python}
entonces la entrada correspondiente de \inl{a} \emph{<también cambia!} Este fenómeno también funciona con listas:
\begin{python}
l=[1,2,3]; k=l; k[1] = 10
\end{python}
En general, en Python las variables son \defn{nombres} de objetos; al poner \inl{b = a}, tenemos <un mismo objeto con dos nombres!

\section{Matrices}
Las matrices se tratan como vectores de vectores, o listas de listas:
\begin{python}
M = array( [ [1, 2], [3, 4] ] )
\end{python}
La forma de la matriz se puede ver con
\begin{python}
M.shape
\end{python}
y se puede manipular con
\begin{python}
M.shape = (4, 1); print M
\end{python}
o con
\begin{python}
M.reshape( 2, 2 )
\end{python}
De hecho, eso es una manera útil de crear las matrices:
\begin{python}
M = r_[0:4].reshape(2,2)
\end{python}

Podemos crear una matriz desde una función:
\begin{python}
def f(i, j):
  return i+j
M = fromfunction(f, (3, 3))
\end{python}




Dado que las matrices son vectores de vectores, al hacer
\begin{python}
print M[0]
\end{python}
nos regresa la primera componente de \inl{M}, que es justamente un \emph{vector}, viz.~el primer renglón de \inl{M}.
Si queremos cierta entrada de la matriz, entonces más bien necesitamos especificar dos coordenadas:
\begin{python}
M[0][1]
M[0, 1]	
\end{python}
[También podemos poner \inl{M.item(1)} que aparentemente es maś eficiente.]

Para extraer ciertas renglones o columnas de \inl{M}, utilizamos una extensión de la notación para vectores:
\begin{python}
M = identity(10)	 	# matriz identidad de 10x10
M[3:5]
M[:, 3:5]
M[3:9, 3:5]
\end{python}


Una función poderosa para construir matrices repetidas es \inl{tile}
\begin{python}
tile( M, (2,2) )
\end{python}

Otros métodos útiles son \inl{diagonal}, que regresa una diagonal de un arreglo:
\begin{python}
diagonal(M)
diagonal(M, 1)
\end{python}
y \inl{diag}, que construye una matriz con el vector dado como diagonal:
\begin{python}
diag([1,2,3])
diag([1,2,3], 2)
\end{python}



\section{Álgebra lineal}
Adentro de \inl{numpy} hay un módulo de álgebra lineal que permite hacer las operaciones más frecuentes que involucran matrices y vectores.

Para empezar, todavía no adentro del módulo \inl{numpy}, están  los productos de una matriz \inl{M} con un vector \inl{v} y de dos matrices. Los dos se llevan a cabo con \inl{dot}:
\begin{python}
M = r_[0:4].reshape(2,2)
v = r_[3, 5]
dot(M, v)
dot(M, M)
\end{python}

\ej Utiliza el \emph{método de potencias} para calcular el vector propio de una matriz cuadrada $M$ correspondiente al valor propio de mayor módulo. Este método consiste en considerar la iteración $M^n \cdot v$.

Para álgebra lineal, se ocupa el módulo \inl{linalg}:
\begin{python}
from numpy import linalg
\end{python}
Entonces las funciones del módulo se llaman e.g.\ \inl{norm} para la norma de un vector se llama
\begin{python}
linalg.norm(v)
\end{python}
Una alternativa es importar todas las funciones de este submódulo
\begin{python}
from numpy.linalg import *
\end{python}
entonces ya no se pone \inl{linalg.}, y se puede referir simplemente a \inl{norm}.

Algunas de las funciones útiles incluyen \inl{linalg.eig} para calcular los valores y vectores propios de una matrix:
\begin{python}
linalg.eig(M)
\end{python}
y \inl{linalg.eigvals} para solamente los valores propios. Hay versiones especiales \inl{eigh} y \inl{eigvalsh} para matrices hermitianas o simétricas reales.

También hay \inl{det} para determinante, e \inl{inv} para la inversa de una matriz.
Finalmente, se puede resolver un sistema de ecuaciones lineales $M \cdot x = b$ con
\begin{python}
linalg.solve(M, b)
\end{python}

\section{Funciones para resumir información sobre vectores y matrices}
Hay varias funciones que regresan información resumida sobre vectores y matrices, por ejemplo
\begin{python}
a = r_[0:16].reshape(4,4)

a.max()
a.min(1)	# actua sobre solamente un eje y regresa un vector
a.mean(0)  	
a.mean(1)
a.sum(1)
\end{python}

También podemos seleccionar a los elementos de un arreglo que satisfacen cierta propiedad:
\begin{python}
a > 5
a[a > 5] = 0
\end{python}

Además existe una función \inl{where}, que es como una versión vectorizada de \inl{if}:
\begin{python}
b = r_[0:16].reshape(4,4)
c = list(b.flatten())
c.reverse()
c = array(c).reshape(4,4)

a = where(b < 5, b, c)
\end{python}
Este comando pone cada entrada de \inl{a} igual a la entrada correspondiente de \inl{b} si ésta es menor que $5$, o a la de \inl{c} si no.







\section{Números aleatorios}

La biblioteca \inl{numpy} incluye un módulo amplio para manipular números aleatorios, llamado \inl{random}.
Como siempre, las funciones se llaman, por ejemplo, \inl{random.random()}. Para facilitarnos la vida, podemos importar todas estas funciones al espacio de nombres con
\begin{python}
from numpy import *
random?	# informacion sobre el modulo
from random import *		# 'random' esta
\end{python}
o
\begin{python}
from numpy.random import *
\end{python}
Nótese que hay otro módulo \inl{random} que existe afuera de \inl{numpy}, con distinta funcionalidad.

La funcionalidad básica del módulo es la de generar números aleatorios distribuidos de manera uniforme en el intervalo $[0,1)$:
\begin{python}
random()
for i in xrange(10): 
  random()
\end{python}
Al poner \inl{random(N)}, nos regresa un vector de números aleatorios de longitud \inl{N}.

Para generar una matriz aleatoria, podemos utilizar
\begin{python}
rand(10, 20)
\end{python}
Para números en un cierto rango $[a, b)$, podemos utilizar 
\begin{python}
uniform(-5, 5, 10)
\end{python}


También hay diversas funciones para generar números aleatorios con distribuciones no-uniformes, por ejemplo \inl{exponential(10)} y \inl{randn(10, 5, 10)} para una distribución normal, o \inl{binomial(10, 3, 100)}.





\ej Calcula la distribución de distancias entre valores propios consecutivos de una matriz aleatoria real y simétrica.


\section{Transformadas de Fourier}
Otro submódulo útil de \inl{numpy} es \inl{fft}, que provee transformadas rápidas de Fourier:
\begin{python}
from numpy import *

x = arange(1024)
f = fft.fft(x)
y = fft.ifft(f)
linalg.norm( x - y )
\end{python}










 
















