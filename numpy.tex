\chapter{Operaciones matemáticas y la biblioteca de arreglos \texttt{numpy}}

\section{Conversión de números}
Acordémonos que hay distintos tipos de números en Python, principalmente enteros (\inl{int}) y flotantes (\inl{float}).
Para convertir entre diferentes tipos, incluyendo cadenas, podemos utilizar
\begin{python}
a = float(3)
b = float('3.5')
\end{python}


\section{Aritmética con precisión arbitraria}

A veces, es necesaria poder llevar a cabo operaciones aritméticas con números flotantes (reales) con precisión superior a los 16 dígitos que provee el \inl{float} (número de ``doble precisión'') de Python. Para hacerlo, existen varios proyectos que proveen bibliotecas con este fin.
La mayoría de estas bibliotecas son interfaces a otros proyectos escritos en C++.

Aquí veremos una opción, la biblioteca \inl{mpmath}, que está escrito completamente en Python. En principio eso lo hace más lento, pero más fácil de entender y modificar el código.

Para cargar la biblioteca, hacemos
\begin{python}
from mpmath import *
\end{python}
Para cambiar la precisión, hacemos
\begin{python}
mp.dps = 50
\end{python}
Ahora, para crear un número flotante de esta precisión, hacemos
\begin{python}
x = mpf('1.0')
\end{python}
donde el número se expresa como cadena.

Al hacer manipulaciones con \inl{x}, los cálculos se llevan a cabo en precisión múltiple.
Por ejemplo,
\begin{python}
print x/6., x*10
print mpf('2.0')**2**2**2**2
\end{python}
Con \inl{mpmath}, no hay límite del exponente que se puede manejar.
También están definidas muchas funciones, por ejemplo \inl{sin}, \inl{exp} y \inl{log}.

Para imprimir un número con una precisión dada, usamos
\begin{python}
nprint(x, 20)
\end{python}



